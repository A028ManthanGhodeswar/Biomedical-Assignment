\documentclass[12pt]{article}
\usepackage{setspace}
\usepackage{multicol}
\usepackage{hyperref}
\hypersetup{colorlinks = true, citecolor = blue, linkcolor = blue, urlcolor = blue}
\usepackage{mathptmx}

\usepackage{setspace}
\usepackage{graphicx}
\graphicspath{{images}}

\title{National Institute of Technology, Raipur}

\begin{document}
\maketitle
\begin{figure}[h]
\centering
\includegraphics[scale=0.50]{Nitrr.png}
\end{figure}
\bigskip
\bigskip
\centering
\begin{Large}
\title{Biomedical Engineering Assignment 3}
\end{Large}

\bigskip
\bigskip
\bigskip
\bigskip

\raggedright 
\begin{multicols}{2}
\author{Submitted By : Manthan Ajit Ghodeswar
\linebreak Roll no:- 21111028,
\linebreak Branch :- Biomedical Branch
\linebreak Semester :- First(2021-22)
\linebreak NIT Raipur, Chattisgarh}
\columnbreak
\columnbreak


Under the supervision of
\linebreak Dr.Saurabh Gupta
\linebreak Department of Biomedical Engineering,
\linebreak NIT Raipur, Chattisgarh
\end{multicols}


\clearpage

\centering
\begin{Huge}
THE FUTURE OF HEALTH CARE
\end{Huge}
\linebreak
\linebreak
\linebreak

\raggedright
\begin{normalsize}
The future of health care is digital. Digital health has been experiencing a great rise as the consumers over the world have been prioritizing not only physical but mental health as well. The demands for flexible, portable devices has been increasing significantly and is being met with very promising technology. Let's take for instance Sleep Number's innovative smart bed technology that uses IoT architecture, machine learning and 3D smart fabric automatically adapts to the individual's needs helping in activating our body's best defense immune system by providing quality sleep.
\linebreak
\linebreak
Technology and health care go hand in hand. Here are some technologies that seem promising and vital in their contribution to health care:-
\linebreak
\linebreak
\begin{large}
1. Nanorobots :-
\end{large}
Nanotechnology which seems very promising, could actually be very close to be available and applicable. Researchers in US and South Korea have already created nanorobots capable of delivering drugs to clogged  arteries and drilling through them. Although there seems to be some issues which will need resolving before it can be applied to humans. Hopefully, we will have it up and running in the very near future.

\bigskip
\begin{large}
2. Artificial Intelligence (AI) :-
\end{large}
AI would be really useful in data interpretation, organization and accessing files. It could also help in identifying potential risks and reduce expenditure significantly. It could also be applied in reducing waste and expediting the drug delivery process.

\bigskip
\begin{large}
3.Patient Access Solutions :-
\end{large}
One of the major areas were problems arise is the health care and billing section. Frustration is very common while chasing down people and could lead to mistakes. Patient Access Solutions would make the whole process simpler and the audit more efficient.

\bigskip
\begin{large}
4. Augmented Reality (AR) :-
\end{large}
Another field which seems to be very promising in health care. It would contribute in keeping the information more organized while avoiding errors and improving the quality of health care. It would also make it possible to access patient information during an interaction, making it more personal and powerful.

\bigskip
\begin{large}
5. 3D Printing :-
\end{large}
3D printing literally has the potential to revolutionize our current health care systems, from prosthetics to instrumentation and even implants.

\bigskip
\begin{large}
6. Shockwave Therapy :-
\end{large}
Shockwave therapy is performed to increase the blood flow to the male genital organ permanently. This is already being used in clinics. However, a new device, called the Phoenix, would allow men to improve the blood flow at home and giving them privacy.

\bigskip
\begin{large}
7. Cyborgization :-
\end{large}
This is a concept that allows humans and machines to work together better and seamlessly in many contexts. This technology would help providers to precisely control robotic surgical tools and enable patients to have integrated systems that would monitor vitals and alert when needed.

\bigskip
\begin{large}
8.e-Prescribing :-
\end{large}
Electronic prescription is really helpful in speeding up the process while reducing travel time and costs and any further discomfort.

\bigskip
\begin{large}
9.Digital Resources for Diagnosis :-
\end{large}
Digital diagnostic tools are becoming more and more powerful. They will help speeding the process via e-prescription and could also help in getting a second opinion and confirm a difficult diagnosis.

\bigskip
\begin{large}
10. Patient Portals :-
\end{large}
Storage as an issue has been countered well since health care went digital. But organizing it is still an issue which could be eased by having patient portals that would enable to access all the patient's information and medical history from one place.

\bigskip
\begin{large}
11. Blockchain :-
\end{large}
Blockchain is and will be playing a crucial part in protecting patient's private information and security purposes. The ledger technology facilitates the secure transfer of patient medical records, manages the medicine supply chain and helps health care researchers unlock genetic code.

\bigskip
\begin{large}
12. Cognitive Technology :-
\end{large}
Cognitive technology can be used to identify and analyze patterns that can be used for predicting early diseases and help catch it before it happens. This technology can prove to be extremely useful in the case of life-threatening disease like cancer. It can also improve patient care by linking patients to their data rather than to their identities.
\end{normalsize}
\clearpage

\begin{Large}
References
\end{Large}

\bigskip

1. $https://www.ces.tech/Articles/2021/October/The-Future-of-Healthcare-is-Digital.aspx$
\linebreak
\linebreak

2. $https://southernmarylandchronicle.com/2021/12/28/12-technology-innovations-that-will-influence-the-future-of-healthcare/$

\end{document}