\documentclass[12pt]{article}
\usepackage{setspace}
\usepackage{multicol}
\usepackage{hyperref}
\hypersetup
{colorlinks = true, citecolor = blue, linkcolor = blue, urlcolor = blue}
\usepackage{mathptmx}
\usepackage{graphicx}
\graphicspath{{Images}}
\title{National Institute of Technology, Raipur}

\begin{document}
\maketitle
\begin{figure}[h]
\centering
\includegraphics[scale=0.50]{NITRR.jpg}
\end{figure}
\bigskip
\bigskip
\centering
\begin{Large}
\title{Biomedical Engineering Assignment 1}
\end{Large}

\bigskip
\bigskip
\bigskip
\bigskip

\raggedright
\begin{multicols}{2}
\author{Submitted By : Manthan Ajit Ghodeswar
\linebreak Smemester :- Third (2022)
\linebreak Nit Raipur, Chhattisgarh}
\columnbreak
\columnbreak


Under the supervision of 
\linebreak Dr. Saurabh Gupta
\linebreak Department of Biomedical Engineering,
\linebreak NIT Raipur, Chhattisgarh
\end{multicols}

\clearpage

\centering
\tableofcontents
\clearpage

\begin{LARGE}
\centering
AN INTRODUCTION TO THE HUMAN 
\linebreak 

BODY (SUMMARY)
\end{LARGE}

\raggedright
\begin{small}
\section{Basic Terminologies :-}
\raggedright
\textbf{Anatomy :-} Anatomy (ana- =up; -tomy = process of cutting) is the science of body structures and the relationships among them. It was first studied by dissection.
\linebreak
\textbf{Physiology :-} Physiology (physio- = nature; -logy = study of) is the science of body functions - how the body parts work).
\linebreak
\textbf{Disorder :-} A disorder is any abnormality of structure or function.
\linebreak
\textbf{Disease :-} Disease is a more specific term for an illness characterized by a recognizable set of signs and symptoms.
\linebreak
\textbf{Signs and Symptoms :-} Objective changes observable by a clinician are called signs whereas symptoms are unapparent subjective changes in body functions.

\section{Levels of Organization and Body Systems :-}
The human body has six levels of organization. \textbf{(a) Chemical Level} (Letters), \textbf{(b) Cellular Level} (Words), \textbf{(c) Tissue Level} (Sentences), \textbf{(d) Organ Level} (Paragraphs), \textbf{(e) System Level)} (Chapter) and \textbf{(f) Organism Level} (Book).
The human body consists of 11 systems including \textbf{(a) Inegumentary}, \textbf{(b) Skeletal}, \textbf{(c) Muscular},\textbf{(d)Nervous}, \textbf{(e)Endocrine}, \textbf{(f)Cardiovascular}, \textbf{(g)Lymphatic and Immunity}, \textbf{(h)Respiratory}, \textbf{(i)Digestive}, \textbf{(j)Urinary} and \textbf{(k)Reproductive} systems.

\section{Basic Life Processes :-}
There are six basic life processes including - \textbf{(a) Metabolism} :- Catabolism (catabol- =
throwing down;-ism = a condition) + Anabolism (anabol- = raising up); \textbf{(b)
Responsiveness} :- Ability to detect and respond to changes; \textbf{(c) Movement} :- Motion of the whole body; \textbf{(d) Growth} :- Increase in body size; \textbf{(e) Differentiation} :- Development of cell from unspecialized to specialized state; and, \textbf{(f) Reproduction} :- Formation of new cells or production of a new individual.

\section{Homeostasis :- }
Homeostasis (homeo- = sameness; -stasis = standing still) is the maintenance of
relatively stable conditions in the body’s internal environment. Homeostasis includes maintaining the volume and composition of body fluids. Fluids within cells is called intracellular fluid, that between the cells is called interstitial fluid,and outside the cell is called extracellular fluid (ECF). ECF within blood vessels is called blood plasma, within lymphatic vessels is called lymph, in brain and around spinal cord is called cerebrospinal fluid, in joints its synovial fluid and in eyes it is aqueous humor and vitreous body. Homeostasis in the human body is continually being disturbed. The disruptions may be from internal environment as well as external.

\section{Feedback Systems :-}
A feedback system (loop) is a cycle of events in which the status of a body condition is monitored, evaluated, changed, re-monitored, re-evaluated. A feedback system includes three basic components.
\textbf{(a) Receptor :-} A body receptor that monitors changes in a controlled condition and
sends input via afferent pathway.
\textbf{(b) Control Center :-} Evaluates the input received from receptors and generates output as nerve impulses and chemical signals sent via efferent pathway.
\textbf{C) Effector :-} Body structure that receives output from the control center and produces a response or effect.

\subsection{Types of Feedback System :- }
\textbf{(a) Negative Feedback System :-} A negative feedback system reverses a change in a
controlled condition. It regulates conditions in the body that remain fairly stable over
long periods. The action of the system slows and then stops as the controlled condition
returns to its normal state.
\textbf{(b) Positive Feedback System :-} A positive feedback system tends to strengthen or
reinforce a change in one of the body’s controlled condition. Some event outside the
system must shut it off. If the action is not stopped, it can ”run away” and be lethal.

\section{Medical Imaging :- }
Refers to techniques and procedures used to create images of the human body.(better
visualisation for diagnosing anatomical and physiological abnormalities and deviations).
Medical imaging includes :- \textbf{Radiography, Magnetic Resonance Imaging (MRI),
Computed Tomography (CT, Ultrasound Scanning, Coronary (Cardiac) Computed
Tomography Angiography (CCTA) Scan, Positron Emission Tomography (PET),
Endoscopy and Radionuclide Scanning.}
\end{small}
\end{document}