\documentclass[12pt]{article}
\usepackage{setspace}
\usepackage{multicol}
\usepackage{hyperref}
\hypersetup{colorlinks = true, citecolor = blue, linkcolor = blue, urlcolor = blue}
\usepackage{mathptmx}

\usepackage{setspace}
\usepackage{graphicx}
\graphicspath{{images}}

\title{National Institute of Technology, Raipur}

\begin{document}
\maketitle
\begin{figure}[h]
\centering
\includegraphics[scale=0.50]{Nitrr.png}
\end{figure}
\bigskip
\bigskip
\centering
\begin{Large}
\title{Biomedical Engineering Assignment 4}
\end{Large}

\bigskip
\bigskip
\bigskip
\bigskip

\raggedright 
\begin{multicols}{2}
\author{Submitted By : Manthan Ajit Ghodeswar
\linebreak Roll no:- 21111028,
\linebreak Branch :- Biomedical Branch
\linebreak Semester :- First(2021-22)
\linebreak NIT Raipur, Chattisgarh}
\columnbreak
\columnbreak


Under the supervision of
\linebreak Dr.Saurabh Gupta
\linebreak Department of Biomedical Engineering,
\linebreak NIT Raipur, Chattisgarh
\end{multicols}


\clearpage

\begin{Huge}
Disruptive Innovations in Health Care
\end{Huge}
\linebreak
\linebreak
\linebreak
\begin{normalsize}
Disruptive innovations do not bring about just a change, they bring about a transition. They change the way we approach having a rippling effect throughout the industry. It is probably no surprise to see technology being the major factor behind these impacting and groundbreaking innovations.
\linebreak

Here are some of the innovations that can be classified disruptive as of now :-
\linebreak
\linebreak
\raggedright
\begin{Large}
1.Consumer Devices, Wearables, and Apps :-
\end{Large}
\linebreak
In the past, people had to often visit the doctor for accessing their biometric data that being vitals such as pulse, heart rate, blood oxygen level and blood pressure etc. Now, because of the inventions of Fitbits, smartwatches and mobile fitness apps one can monitor his vitals in his home anytime he wants thus saving time and money and can even easily share his data with the doctor in a matter of seconds.
\end{normalsize}
\linebreak
\linebreak
\begin{Large}
2.AI and ML :-
\end{Large}
\linebreak
AI can prove to be of great use when it comes to the interaction with patient and collecting and sorting data. It would also be capable of billing. It would also help in bringing down health care costs and let doctor focus on the patient more than the documentation procedure. And machine learning will help improve the accuracy as and when it starts collecting and processing more data.
\linebreak
\linebreak
\begin{Large}
3. Blockchain :-
\end{Large}
\linebreak
Blockchain is a database technology which uses encryption and other security measures that is also capable of linking data in such a way that would allow access to it only to the authorized thus providing privacy and security to not only patient records but also to the supply and distribution data of the hospitals. 
\linebreak
\linebreak
\begin{Large}
4.IoT :-
\end{Large}
Internet of things (IoT) will play a vital role in helping the people share their data and record gathered on the smart wearables helping in quick access. Information being shared so quickly will also help the patient in getting a second opinion if required or needed very quickly.
\linebreak
\linebreak
\clearpage


\begin{Large}
5. Electronic Health Records (EHRs) and big data :-
\end{Large}
Since health care went digital, keeping track of patient records electronically has addressed accessibility issues and helped in keeping the data more organized. The large amounts of data already collected could be analyzed with AI and ML and would be useful in linking the patients with the same data thus reducing time to administer the treatment and with better accuracy too.
\linebreak
\linebreak
\begin{Large}
6. Telemedicine :-
\end{Large}
\linebreak
Telemedicine is here to stay and quite rightly so. Since the COVID pandemic the need of the hour was fulfilled by telemedicine helping doctors communicate and try to diagnose the patient virtually. It would also save time and effort to not visit the doctor every time. Although, it is highly dependent on the internet and things could go south when there's any connectivity issue.
\linebreak
\linebreak
\begin{Large}
7. Patient Rights :-
\linebreak
\end{Large}
Under the Right to Information Act (2005), patient has the right to access his medical records and even the billing process as to where his money is being spent. The hospitals have also been asked to be more transparent with their data to the patient.
\linebreak
\linebreak
\begin{LARGE}
CONCLUSION :-
\end{LARGE}
\linebreak
\linebreak
Disruptive innovation has spread over the health care industry, but there is still ambiguity on the usage of the term. That's probably because it has already set such high standards that the expectations are certainly much higher. If these innovations aren't identified properly, it could lead to the invention not reaching its intended full potential. This would lead in delay of the product being accessible, and even if it were to be accessible it wouldn't be economically feasible. There should be a more precise health care specific definition or a certain standard or a benchmark or some sort of certification through an expert consensus process that would act as a precursor helping in better identification of potentially beneficial disruptive innovations and will help in making the accessible more quickly at an economical and reasonable rate to the locals. An innovation can't be disruptive if not available and accessible reasonably.
\linebreak
\linebreak
\clearpage

\begin{Large}
References :-
\end{Large}
\linebreak
\linebreak
\begin{small}
 $https://dhge.org/about-us/blog/disruptive-innovation-healthcare-examples$
\linebreak
\linebreak
 $https://innovations.bmj.com/content/7/1/208$
\end{small}

\end{document}