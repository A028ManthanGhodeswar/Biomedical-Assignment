\documentclass[12pt]{article}
\usepackage{setspace}
\usepackage{hyperref}
\hypersetup{colorlinks = true, citecolor = blue, linkcolor = blue, urlcolor = blue}
\usepackage{mathptmx}

\usepackage{setspace}
\usepackage{graphicx}
\graphicspath{{images}}

\title{National Institute of Technology, Raipur}

\begin{document} 
\maketitle
\begin{figure}[h]
\centering
\includegraphics[scale=0.50]{Nitrr.png}
\end{figure}
\bigskip
\bigskip
\centering
\begin{Large}
\title{Biomedical Engineering Assignment 1}
\end{Large}

\bigskip
\bigskip
\bigskip
\bigskip

\raggedright 
\author{Submitted By : Manthan Ajit Ghodeswar
\linebreak Roll no:- 21111028,
\linebreak Branch :- Biomedical Branch
\linebreak Semester :- First(2021-22)
\linebreak NIT Raipur, Chattisgarh}



\clearpage

\centering
\tableofcontents
\clearpage

\begin{normalsize}

\section{Artificial Heart Valve}
An artificial heart valve is a one-way valve implanted into a person's heart to replace a valve that is not functioning properly (valvular heart disease). Artificial heart valves can be separated into three broad classes: mechanical heart valves, bioprosthetic tissue valves and engineered tissue valves.
\linebreak
\linebreak
The human heart contains four valves: tricuspid valve, pulmonary valve, mitral valve and aortic valve. Their main purpose is to keep blood flowing in the proper direction through the heart, and from the heart into the major blood vessels connected to it (the pulmonary artery and aorta). Heart valves can malfunction for a variety of reasons, which can impede the flow of blood through the valve (stenosis) and/or let blood flow backwards through the valve (regurgitation). Both processes put strain on the heart and may lead to serious problems, including heart failure. While some dysfunctional valves can be treated with drugs or repaired, others need to be repaired with an artificial valve.
\linebreak
\linebreak
\begin{figure}[h]
\centering
\includegraphics[scale=0.75]{280px-Mechanical_heart_valves.jpg}
\caption{Different Types of Artificial Heart Valves}

\end{figure}


\begin{large}
\subsection{Types of Artificial Valves}
\end{large}

\raggedright
\subsection{Mechanical Valves}
Mechanical valves come in three main types - caged ball, tilting - disc and bileaflet - with various modifications on these designs. Caged ball valves are no longer implanted. Bileaflet valves are the most common type of mechanical valves implanted in patients today.

\subsubsection{Caged Ball Valves}
The first artificial heart valve was the cage ball valve, a type of ball check valve, in which a ball is housed inside a cage. When the heart contracts and the blood pressure in the chamber of the heart exceeds the pressure on the outside of the chamber, the ball is pushed against the cage and allows blood to flow. When the heart finishes contracting, the pressure inside the chamber drops and the ball moves back against the base of the valve forming a seal.
\begin{figure}[h]
\centering
\includegraphics[scale=0.080]{Caged Ball Valve.jpg}
\caption{Caged Ball Valve}
\end{figure}

\subsubsection{Tilting-Disc Valves}
Tilting-disc valves, a type of swing check valve, are made of a metal ring covered by an ePTFE(Polytetrafluoroethylene) fabric. The metal ring holds, by means of two metal supports, a disc that opens when the heart beats to let blood flow through, then closes again to prevent blood flowing backwards. The disc is usually made of an extremely hard carbon material (pyrolytic carbon), enabling the valve to function for years without wearing out.
\begin{figure}[h]
\centering
\includegraphics[scale=0.050]{Chitra_Valve.jpg}
\caption{Tilting-disc Valve}
\end{figure}
\subsubsection{Bileaflet Valves}
Bileaflet valves are made of two semi-circular leaflets that revolve around struts attached to the valve housing. With a larger opening than caged ball or tilting-disc valves, they carry a lower risk of blood clots. They are, however, vulnerable to blood backflow.
\begin{figure}[h]
\centering
\includegraphics[scale=0.20]{leaflet valve.jpg}
\caption{Bileaflet Valve}
\end{figure}
\subsection{Bioprosthetic Tissue Valves}
Bioprosthetic valves are usually made from animal tissue(heterograft/xenograft) attached to a metal or a polymer support. Bovine (cow) tissue is most commonly used, but some are made from porcine (pig) tissue. This tissue is treated to prevent rejection and calcification.Alternatives to animal tissue valves are sometimes used, where valves are used from human donors, as in aortic homografts and pulmonary autografts. An aortic homograft is an aortic valve from a human donor, retrieved either after their death or from a heart that is removed to be replaced during a transplant. A pulmonary autograft, also known as the Ross procedure, is where the aortic valve is removed and replaced with the patient's own pulmonary valve.

\subsection{Tissue-Engineered Valves}
For over 30 years researchers have been trying to grow heart valves in vitro. These tissue-engineered valves involve seeding human cells on to a scaffold. There are two main types of scaffold. First are the natural ones like the decellularised tissue and the other are the ones made from degradable polymers. Scaffold helps in guiding tissue growth into correct 3-D structure of the heart valve. Unfortunately, tissue-engineered heart valves aren't commercially available to us yet.
\linebreak
\linebreak
The advantage is that they can be person-specific, which is a big leap for us when it comes to personalized healthcare or precision medicine. They can be 3-D modeled to fit an individual recipient. It uses 3-D printing to maintain high accuracy and precision as it has to deal with different biomaterials. The cells used for this procedure are expected to secrete extracellular matrix(ECM). The matrix provides support in maintaining shape and determining cell activities.

\subsection{Differences Between Mechanical and Tissue Valves}
Both are made up of different materials. Mechanical valves are made of titanium and carbon whereas tissue valves constitute human or animal tissue. Mechanical valves can be a better choice when it comes to younger people and people at risk due to its durability. It can also be preferable for those who are already taking blood thinners and the ones who wouldn't want to consider another surgery of valve replacement.
\linebreak
\linebreak
Considering the risk of blood clots forming for mechanical valves and severe bleeding being a major downgrade of taking blood-thinning medications, tissue valves do become preferable in such circumstances. It can also be preferred by people who are not willing to take warfarin, which also happens to cause risk in pregnancy.

\centering
\section{Electron Microscope}
An electron microscope is a type of microscope that applies a beam of accelerated electrons as a source of illumination. As we know, the wavelength of electron can be incredibly small when compared to that of visible light photons, electron microscopes have a high resolving power, much higher than light microscopes and therefore can reveal the structure of smaller objects.
Electron microscopes are used to investigate the structure of microorganisms, cells, large molecules, biopsy samples, metals, and crystals. Industrially, they are often used for quality control and failure analysis.
\begin{figure}[h]
\centering
\includegraphics[scale=0.40]{Electron_Microscope}
\caption{Modern Electron Microscope}
\end{figure}

\subsection{Types of Electron Microscope}

\raggedright
\subsubsection{Transmission Electron Microscope (TEM)}
The transmission electron microscope uses a high voltage electron beam to illuminate the specimen and create an image. The beam is produce by an electron gun, commonly fitted with a tungsten filament cathode as the electron source.One major disadvantage of this microscope is the need for extremely thin sections of specimen ranging to about a 100 nanometers which makes the specimen creating part technically very very challenging.

\subsubsection{Serial-section Electron Microscope (ssEM)}
It is basically a version based on an application of transmission electron microscope. Specifically, in analyzing the connectivity in volumetric samples of brain tissue by imaging many thin sections in sequence. This is done by introducing a milling method into the imaging pipeline, by which successive slices of 3D volume are exposed to the beam and imaged.

\subsubsection{Scanning Transmission Electron Microscope(STEM)}
It focuses a incident probe across a specimen which is further thinned to facilitate the detection of electrons scattered through the specimen. Thus, it can be concluded, that it is basically a higher resolution version of the transmission electron microscope. There is one major difference though. In the case of STEM, the focusing action occurs before the electrons hit the specimen, which in case of TEM, happens afterwards.

\subsubsection{Scanning Electron Microscope (SEM)}
Generally, the image resolution in SEM is lower than that of a TEM. However, SEM images the surface of a sample and not the interior, meaning, the electrons do not have to travel through the sample. Thus reducing efforts and the need for extensive sample preparation.One of the plus points of the SEM is that it is able to image bulk samples which can fit on its stage and still be maneuvered. It also has a great depth of field and can produce images which are good representations of the 3-D surface shape of the sample.

\subsubsection{Reflection Electron Microscope (REM)}
In the case of reflection electron microscope, just as in the TEM, an electron beam is incident on a surface but instead of using the transmission(TEM) or secondary electrons(SEM) the reflected beam of elastically scattered electrons is detected.

\subsubsection{Scanning Tunneling Microscopy (STM)}
In STM, a conductive tip held at a voltage is brought near a surface, and a profile can be obtained based on the tunneling probability of an electron from the tip to the sample since it is a function of distance.

\subsection{Color}
Commonly, electron microscopes produce images with a single brightness value per pixel, with the results usually rendered in greyscale. However, these images can be colorized through feature-detection software or by hand editing using a graphics editor for better visual and aesthetic effects. Although generally, this does not add any new information about the specimen.
\linebreak
\linebreak
Some of the detectors used in SEM have analytical capabilities, and can they provide multiple items of data at each pixel. Examples include the energy-dispersive X-ray spectroscopy (EDS) detectors used in elemental analysis and cathodoluminescence microscope (CL) systems that analyze the intensity and spectrum of electron - induced luminescence.

\subsection{Disadvantages}
Electron microscopes are expensive to build and maintain. The microscopes designed to achieve high resolutions must be kept in stable buildings sometimes even underground to ensure the cancellation of magnetic fields.The samples have to be largely viewed in vacuum, as the molecules that make up air would result in scattering of electrons.Scanning electron microscopes operating in conventional high-vacuum mode usually image conductive specimens; therefore for non-conductive specimens, a conductive coating is needed. This coating could be of gold/palladium alloy, carbon, osmium,etc. 
\linebreak
\linebreak
Samples of hydrated materials, including almost all biological specimens have to be prepared in various ways to stabilize them, reducing their thickness (ultrathin sectioning) and increase their electron optical contrast (staining). So, the setup process could prove to be demanding and time consuming in such cases. 
\clearpage

\section{Automated External Defibrillator}
An automated external defibrillator(AED) is a type of a portable electronic device that helps in automatically diagnosing the life-threatening cardiac arrhythmias of ventricular fibrillation(VF) and pulses ventricular tachycardia. It treats them through defibrillation, the application of electricity which stops the arrhythmia, allowing the heart to re-establish an effective rhythm. Not just portable, with simple audio and visual commands, AEDs are very simple to use for the layperson. Also, use of AEDs is taught in many first aid, certified first responder, and basic life support (BLS) level cardiopulmonary resucitation(CPR) classes.

\subsection{Conditions that the device treats}
As mentioned, an AED is used in case of life threatening cardiac arrhythmias which lead to sudden cardiac arrest. In shockable cardiac arrhythmia, the heart is electrically active but in a dysfunctional pattern that doesn't allow it to pump and circulate blood. In ventricular tachycardia, the heart beats to fast to effectively pump blood ultimately leading to ventricular fibrillation. In this case, the electrical activity of the heart will turn chaotic, preventing the ventricle from effectively pumping blood. Thus, decreasing fibrillation in heart over time and eventually reaching asystole.

\subsection{Requirements for use}
AEDs are designed to be used by laypersons who ideally should have received AED training. Bras with a metal underwire and piercings on the torso must be removed before using the AED on someone to avoid interference. In a study analyzing the effects of AEDs immediately present during Chicago's Heart Start program over a two-year period, of 22 individuals, 18 were in a cardiac arrhythmia which AEDs can treat.Of these 18, 11 survived. And out of these 11, 6 were treated by bystanders with absolutely no previous training in the use of AED.

\subsection{Placement and Availability}
AEDs are generally kept where health professionals and first responders can use them. Public access units can be found in corporate and government offices, shopping centres, restaurants, public transport etc. In order for them to stand out and easily visible, public access AEDs are often brightly colored and are mounted in protective cases. A trend that is developing is the purchase of AEDs to be used in home, particularly by those with known existing heart conditions especially as the prices have fallen to more affordable levels. Although there are rising concerns among the medical professionals as the people at home may not have proper training.
\linebreak
\linebreak
Typically, an AED kit will contain a face shield which acts as a barrier between patient and first aider during rescue breathing; a pair of nitrile rubber gloves; a pair of trauma shears for cutting through a person's clothing and expose the chest; a small towel for wiping away any moisture and a razor in case shaving is needed.
\begin{figure}[h]
\centering
\includegraphics[scale=0.20]{AED.jpg}
\caption{Automatic External Defibrillator}
\end{figure}

\subsection{Preparation for Operation}
Most manufacturers recommend checking the AED before every period of duty or on a regular basis for fixed units. Some units need to be switched on in order to perform a self check; other models have a self check system built in with a visible indicator.
\linebreak
\linebreak
The electrode pads are also marked with an expiration date, and it is important to ensure that the pads are in date. The typical life expectancy of AED pads are between 18 and 30 months. Some models are designed to make this date visible through a window, although others will require to be opened in order to find the date stamp.
\linebreak
\linebreak
It is also important to ensure that the AED unit's batteries have not expired. It is specified by the manufacturer that how often the batteries should be replaced. Each AED has a different recommended maintenance schedule outlined in the user manual. Common checkpoints on every checklist, however, also include a monthly check of the battery power by checking the green indicator light when powered on, condition and cleanliness of all cables and the unit, and check for the adequate supplies.

\subsection{Simplicity of Use}
Unlike regular defibrillators, an AED requires minimal training to be used. In many countries, AEDs approved use an electronic voice to prompt users through each step. In case of the user being hearing impaired, many AEDs now include visual prompts as well. Most units are designed for usage by non-medical operators.
\linebreak
\linebreak
The AED automatically diagnoses the heart rhythm and determines if a shock is needed. Automatic models even have the capability to deliver shock without the user's command. Semi-automatic models will tell the users that a shock is needed, but the user must tell the machine to do so, usually by pressing a button.

\subsection{Benefit}
Observational studies have shown that in out of hospital cardiac arrest, public access defibrillators when used were associated with 40 per cent median survival. When operated by non-dispatched lay first responders they have the highest likelihood of leading to survival.
\clearpage
\centering
\section{Pacemaker}
\raggedright
A pacemaker is a small device that is implanted in the chest to control the heartbeat, mainly, when it is too slow. Implanting it requires the patient to undergo surgical procedure.

\subsection{Types of Pacemaker}

\subsubsection{Single Chamber Pacemaker}
Usually carries electrical signals to the right ventricle of our heart.

\subsubsection{Dual Chamber Pacemaker}
Usually carries electrical signals to the right ventricle and right atrium of our heart. It helps in controlling the timing of contraction between the two chambers.

\subsubsection{Biventricular Pacemaker}
Biventricular pacing, also called resynchronization therapy,, is for people who have heart failure and heartbeat problems. In this case, the pacemaker stimulates both the left and right ventricles to help the heart beat more efficiently.

\subsection{Functioning of a Pacemaker}
Pacemakers work only when needed. If our heartbeat is too slow, it will send electrical signals to our heart to correct the beat. With time, pacemakers have grown having sensors that detect body motion or breathing rate and signal the devices to increase heart rate accordingly during exercise.
\linebreak
\linebreak
A pacemaker has two parts:-
\subsubsection{Pulse Generator}
Here, a small metal container houses a battery and the electrical circuitry that controls the rate of electrical pulses sent to the heart.

\subsubsection{Leads (electrodes)}
One to three flexible, insulated wires are each placed in one or more chambers as needed and deliver the electrical pulses to adjust the heart rate. However, most of the new pacemakers do not require leads and are hence termed as leadless pacemakers and are directly implanted in the heart muscle.
\begin{figure}[h]
\centering
\includegraphics[scale=0.40]{Pacemaker.png}
\caption{Pacemaker}
\end{figure}
\subsection{Risks}
Due to the advancements in modern healthcare, complications due to pacemaker surgery aren't that common. Although, they could include:-
\linebreak
1)A possible infection in the site of the heart where the device is implanted.
\linebreak
2)Swelling, bruising or bleeding at the site of pacemaker, especially if the patient consumes blood thinners.
\linebreak
3)Blood clots (thromboembolism) near the pacemaker site.
\linebreak
4) Damage to blood vessels or nerves near the pacemaker.
\linebreak
5)Collapsed lung (pneumothorax)
\linebreak
6)Blood in the space between the lung and chest wall (hemothorax)
\linebreak
7)Shifting of the device or leads, which could lead to cardiac perforation. Although, very rare.
\subsection{Results}
Having a pacemaker should improve symptoms caused by slow heartbeat which include fatigue, lightheadedness and fainting. Unlike the earlier versions of pacemakers, today's modern versions automatically adjust the heart rate to match the level of patient's physical activity, which allows the patient to resume a more active lifestyle making him more comfortable.
\linebreak
\linebreak
After the patient is implanted with a pacemaker, visiting a doctor after 3-6 months is must. Signs to look out for after implantation are gaining of weight, legs or ankles getting puffy, or fainting or getting dizzy.
\linebreak
\linebreak
However, there is one plus point that the pacemakers nowadays can be checked by the doctors remotely, meaning the patient need not have to go to the doctor's office every now and then. The pacemaker will send information to the doctor, including the patient's heart rate and rhythm, and how the pacemaker is working  and also about the remaining battery life.
\linebreak
\linebreak
Pacemaker's battery should last ranging from 5 - 15 years. A disadvantage is that the patient will have to undergo surgery once again to get the batteries replaced. However, on the brighter side, the procedure is often quicker and requires less recovery time than in the case of implantation.

\subsection{Dealing With End - of - Life Issues}
If the patient has a pacemaker and unfortunately gets terminally ill with a condition unrelated to the heart, like cancer, it is possible that the pacemaker may be able to prolong the patient's life. Although doctors and researchers vary in their opinions whether the pacemaker should be turned off in such end - of - life situations.
\linebreak
Thus in such cases, it may come down to the patient or his family or the person responsible for taking medical decisions for him. 

\subsection{Life Expectancy with Pacemaker}
In a test conducted with 6505 patients and analyzing their follow ups, median survival was 101.9 months which is approximately 8.5 years. Wherein, 44 per cent of patients were alive after 10 years and 21.4 per cent alive after 20 years. It was concluded that survival of patients is independently influence by their baseline characteristics.

\centering
\section{Insulin Pumps}
\raggedright
\subsection{What are Insulin Pumps?}
Everyone with type 1 diabetes and many people with the type 2 need to take insulin to manage their blood sugar level. This can be done either by : injecting it with a pen, or using an insulin pump.
\linebreak
\linebreak
An insulin pump is a small computerized device. It delivers insulin through a thin tube that goes under the patient's skin.
\subsection{Working of an Insulin Pump}
It releases insulin almost the way our body naturally would : a steady flow throughout the day and night, termed as basal insulin, and an extra dose at mealtime, called a bolus, to handle the rising blood sugar from the food consumed. The pump is programmable for both basal and bolus doses depending upon the food consumption level of the patient. For example, if the patient eats more than normal, the pump can program a larger bolus to cover the carbs in his food. It can also bring down high sugar level at other times, too.\linebreak
\linebreak
The pump is relatively ergonomic, about the size of a smartphone. It is to be attached to the body using an infusion set: thin plastic tubing and either a needle or a small tapered tube called the cannula put under the skin. The place where it is putt which may be the belly, buttock or sometimes thigh is called the infusion site. Some pumps come with inserters for easier placement even in hard-to-reach areas. Insulin pumps use short-acting and rapid-acting insulin not the long acting, since it is programmed to continuously deliver a small amount to keep the blood sugar levels even.
\begin{figure}[h]
\centering
\includegraphics[scale=0.20]{Insulin Pump.png}
\caption{Insulin Pump}
\end{figure}
\subsection{Advantages}
1) Fewer needle sticks are needed. A pump requires one shot every few days when infusion set is changed.
\linebreak
2)A pump is actually more accurate than shots, helping in better managing blood sugar levels.
\linebreak
3)The patient will have fewer blood sugar lows, which is important if he often has hypoglycemia.
\linebreak
4)It may improve A1c levels.
\linebreak
5)Dosing for meals and snacks is easier.
\linebreak
6)It's easier to plan for exercise and bolus.
\linebreak
7)It helps in managing early morning high blood sugar, also called the "dawn phenomenon".
\subsection{Disadvantages}
1)Patient needs to enter information in the pump all day and change out the infusion set every few days.
\linebreak
2)Patient has to be committed to using it safely, including checking his blood sugar to cross verify that the pump is working right. Otherwise, he may risk a lethal disease called diabetic ketoacidosis (KDA).
\linebreak
3)Training is needed to learn the usage of the pump, which means several visit to the healthcare teams or full day of outpatient training.
\linebreak
4)Pump supplies are often expensive.

\subsubsection{Furthermore an insulin pump may not be right for the patient if}
1)He doesn't want to let people know that he has diabetes.
\linebreak
2)He doesn't like the feeling of wearing a device.
\linebreak
3)He's not comfortable operating the pump.
\linebreak
4)He doesn't want to check his blood sugar at least four times a day.
\linebreak
5)He isn't sure if he'd be able to work out and keep a track of insulin dosing, carbs and physical activity.
\clearpage

\subsection{Accesibility}
Use of insulin pumps is increasing because of :-
\linebreak
\linebreak
1)Easy delivery of multiple insulin injections for those using intensive insulin therapy.
\linebreak
2)Accurate delivery of small boluses, which can be very helpful when it comes to infants.
\linebreak
3)There is growing support between doctors and insurance companies as the benefits are contributing in reducing any long-term complications.
\linebreak
4)Improvements in blood glucose monitoring. New meters require smaller drops of blood, and the corresponding lancet poke in the fingers is therefore much smaller and less painful. These meters also support alternate site testing for the most routine tests for practically painless testing.

\subsection{Future Developments}
1)The combination of insulin pump technology with a continuous blood glucose monitoring system seems very promising for real time control of the blood sugar level. However, currently, there are no mature algorithms to automatically control the insulin delivery based on the feedback of the blood glucose level.When the loop is closed, the system may function as an artificial pancreas.
\linebreak
2)Dual hormone insulin pumps can either infuse insulin or glucagon. In event of hypoglycemia, the glucagon could be triggered to increase the blood glucose. This would be particularly valuable in a closed loop system under the control of a glucose sensor. The artificial pancreas, is a recently developed device designed with this technology in mind.
\linebreak
3)Ultrafast insulins. As the name suggests, these insulin are absorbed relatively quickly than the currently available Humalog, Novolog and Apidra which have a peak at about an hour. Faster insulin uptake would theoretically help in better coordination with meals, and allow faster recovery from hyperglycemia if the insulin infusion is suspended. However, ultrafast insulins are currently at the development stage.
\clearpage

\section{References}
\begin{small}

1)Artificial Heart Valve - $https://en.wikipedia.org/wiki/Artificial_heart_valve
\linebreak
2)Electron Microscope - https://en.wikipedia.org/wiki/Electron_microscope
\linebreak 
3)AED - https://en.wikipedia.org/wiki/Automated_external_defibrillator
\linebreak
4)Pacemaker - https://www.mayoclinic.org/tests-procedures/pacemaker/about/pac-20384689\#
\linebreak
5)Insulin Pumps - https://www.webmd.com/diabetes/insulin-pump$

\end{small}

\end{normalsize}



\end{document}