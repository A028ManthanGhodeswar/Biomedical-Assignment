\documentclass[12pt]{article}
\usepackage{setspace}
\usepackage{multicol}
\usepackage{hyperref}
\hypersetup{colorlinks = true, citecolor = blue, linkcolor = blue, urlcolor = blue}
\usepackage{mathptmx}

\usepackage{setspace}
\usepackage{graphicx}
\graphicspath{{images}}

\title{National Institute of Technology, Raipur}

\begin{document}
\maketitle
\begin{figure}[h]
\centering
\includegraphics[scale=0.50]{Nitrr.png}
\end{figure}
\bigskip
\bigskip
\centering
\begin{Large}
\title{Biomedical Engineering Assignment 6}
\end{Large}

\bigskip
\bigskip
\bigskip
\bigskip

\raggedright 
\begin{multicols}{2}
\author{Submitted By : Manthan Ajit Ghodeswar
\linebreak Roll no:- 21111028,
\linebreak Branch :- Biomedical Branch
\linebreak Semester :- First(2021-22)
\linebreak NIT Raipur, Chattisgarh}
\columnbreak
\columnbreak


Under the supervision of
\linebreak Dr.Saurabh Gupta
\linebreak Department of Biomedical Engineering,
\linebreak NIT Raipur, Chattisgarh
\end{multicols}


\clearpage

\centering
\tableofcontents
\clearpage

\raggedright
\section{Oxygen :-}
\begin{Large}
As known to us, COVID - 19 patients had reported breathing problems which varied according to the severity of the patients and the seriousness of the infection suffered. Hence the supply of oxygen through nasal cannula or an intrusive face mask was necessary. Oxygen comes in the form of cylinders, small when there's not much demand so as to be more portable and large when needed for stationary patients and longer-term supply. Then there was the alternative of oxygen concentrators possible thanks to the biomedical engineers. They are not typically used in hospital setting but they do have the ability to extract oxygen from the air on demand and ensure direct supply to the patient. It was further made ergonomic as it has size ranging from a portable shoulder bag form to higher stationary machines if the patient needs 24/7 oxygen supply.

\section{Ventilators :-}
Patients who are unable to breathe spontaneously need to be put on ventilators. Modern ventilators are typically closed loop pressure controlled and also capable of detecting spontaneous breathing which synchronizes assistance for recovering patients. Another plus point is that they also enable to control the composition of the gas the patient breathes from normal to 100 per cent oxygen usually using the hospital's gas network, but in case of no network, oxygen gas cylinders and concentrators may also be used.

\section{Personal Protective Equipment :-}
The use of PPE kits and masks remains the most practical line of defense against COVID. Several group of engineers have been developing enhanced PPE including powered air quality purifiers which would be potentially more protective due to the air filtration. Would also avoid mechanical loads for those who have delicate facial skin, be cleanable, reusable and also available at lower cost, even producing less plastic waste than before, so relatively eco - friendly. 

\section{Vaccines and Booster Doses :- }
There have been many variants of COVID. And after extensive research many vaccines were also made and administered to the people. Various vaccines have been introduced to the world which have managed to at least dialed down the infection. Several countries have also started administering needleless vaccines to avoid pricking and administering without any further discomfort. And now even booster doses have been introduced for those already vaccinated as a further measure of precaution.

\section{Wearable Tech and Early Illness Detection :- }
During the pandemic the use of smart wearable has certainly skyrocketed, since now they were made capable of measuring vitals and biometrics. The data recorded by these wearables was enough for researchers to be able to track and detect COVID infections even before people notice they have any symptoms. Although the symptoms detected by these wearables may not always be of COVID and that they may be inaccurate but can still be valued as a preventive measure or sometimes even key to an early and valuable diagnosis.

\end{Large}
\end{document}