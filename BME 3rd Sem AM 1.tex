\documentclass[12pt]{article}
\usepackage{setspace}
\usepackage{multicol}
\usepackage{hyperref}
\hypersetup
{colorlinks = true, citecolor = blue, linkcolor = blue, urlcolor = blue}
\usepackage{mathptmx}
\usepackage{graphicx}
\graphicspath{{Images}}
\title{National Institute of Technology, Raipur}

\begin{document}
\maketitle
\begin{figure}[h]
\centering
\includegraphics[scale=0.50]{NITRR.jpg}
\end{figure}
\bigskip
\bigskip
\centering
\begin{Large}
\title{Biomedical Engineering Assignment 1}
\end{Large}

\bigskip
\bigskip
\bigskip
\bigskip

\raggedright
\begin{multicols}{2}
\author{Submitted By : Manthan Ajit Ghodeswar
\linebreak Smemester :- Third (2022)
\linebreak Nit Raipur, Chhattisgarh}
\columnbreak
\columnbreak


Under the supervision of 
\linebreak Dr. Saurabh Gupta
\linebreak Department of Biomedical Engineering,
\linebreak NIT Raipur, Chhattisgarh
\end{multicols}

\clearpage

\centering
\tableofcontents
\clearpage

\begin{LARGE}
\centering
AN INTRODUCTION TO THE HUMAN 
\linebreak 

BODY (SUMMARY)
\end{LARGE}

\raggedright
\begin{small}
\section{Basic Terminologies :-}
\raggedright
\textbf{Anatomy :-} Anatomy (ana- =up; -tomy = process of cutting) is the science of body structures and the relationships among them. It was first studied by dissection.
\linebreak
\textbf{Physiology :-} Physiology (physio- = nature; -logy = study of) is the science of body functions - how the body parts work).
\linebreak
\textbf{Disorder :-} A disorder is any abnormality of structure or function.
\linebreak
\textbf{Disease :-} Disease is a more specific term for an illness characterized by a recognizable set of signs and symptoms.
\linebreak
\textbf{Signs and Symptoms :-} Objective changes observable by a clinician are called signs whereas symptoms are unapparent subjective changes in body functions.



\section{Levels of Structural Organization and Body Systems :-}
\textbf{1) Chemical Level :-} Begins with the atomic level. (Letters)
\linebreak
\textbf{2)Cellular Level :-} Molecules combine to form cells, the basic structural and functional units of an organism that are composed of chemicals. (Words)
\linebreak
\textbf{3)Tissue Level :-}  Groups of cells and the materials surrounding them that work together to perform a specific functions (Sentences).
\linebreak
\textbf{4) Organ Level :-} Structures that are composed of two or more different types of tissues; they have specific functions and usually recognizable shapes. (Paragraphs)
\linebreak
\textbf{5)System :-} Consists of related organs with a common function. (Chapter)
\linebreak
\textbf{6)Organism Level :-} All the parts of the human body functioning together constitute the total organism. (Book)
\linebreak
\section{Basic Life Processes :-}
There are six basic life processes including - (a) Metabolism :- Catabolism (catabol- = throwing down;-ism = a condition) + Anabolism (anabol- = raising up); (b) Responsiveness :- Ability to detect and respond to changes; (c)Movement :- Motion of the whole body; (d) Growth :- Increase in body size; (e) Differentiation :- Development of cell from unspecialized to specialized state; and, (f) Reproduction :- Formation of new cells or production of a new individual.

\section{Systems of Human Body :-}
The Human Body consists of 11 systems. (a) Integumentary System [Skin and associated structures]; (b) Skeletal System [Bones, Joints, Cartilages]; (c) Muscular System [Skeletal Muscle Tissue]; (d) Nervous System [Brain, Spinal Cord, Nerves, Eyes and Ears]; (e)Endocrine System [Hormone Producing Glands]; (f) Cardiovascular System [Blood, Heart and Blood Vessels]; (g) Lymphatic System and Immunity [Lymphatic fluid and vessels, spleen, lymph nodes, tonsils]; (h) Respiratory System [Lungs, Larynx,Pharynx, Trachea and Bronchial Tubes]; (i) Digestive System [Mouth, Pharynx, Esophagus, Stomach, Small and Large Intestines, Anus, Salivary Glands, Liver, Gallbladder, Pancreas]; (j) Urinary System [Kidneys, Ureters, Urinary Bladder and Urethra]; and, (k) Reproductive System [Gonads and associated organs]. 

\section{Homeostasis :- }
Homeostasis (homeo- = sameness; -stasis = standing still) is the maintenance of relatively stable conditions in the body's internal environment.
\linebreak
Homeostasis includes maintaining the volume and composition of body fluids. Fluids within cells is called intracellular fluid, that between the cells is called interstitial fluid, and outside the cell is called extracellular fluid (ECF). 
\linebreak
ECF within blood vessels is called blood plasma, within lymphatic vessels is called lymph, in brain and around spinal cord is called cerebrospinal fluid, in joints its synovial fluid and in eyes it is aqueous humor and vitreous body.
\linebreak
Homeostasis in the human body is continually being disturbed. The disruptions may be from internal environment as well as external.

\section{Feedback Systems :- }
A feedback system (loop) is a cycle of events in which the status of a body condition is monitored, evaluated, changed, re-monitored, re-evaluated. A feedback system includes three basic components.
\linebreak 
\textbf{(a) Receptor} :- A body receptor that monitors changes in a controlled condition and sends input via afferent pathway.
\linebreak
\textbf{(b) Control Center} :- Evaluates the input received from receptors and generates output as nerve impulses and chemical signals sent via efferent pathway.
\linebreak
\textbf{C) Effector} :- Body structure that receives output from the control center and produces a response or effect.
\subsection{Types of Feedback Systems :- }
\textbf{(a) Negative Feedback System :-} A negative feedback system reverses a change in a controlled condition. It regulates conditions in the body that remain fairly stable over long periods. The action of the system slows and then stops as the controlled condition returns to its normal state.
\linebreak
\textbf{(b) Positive Feedback System :-} A positive feedback system tends to strengthen or reinforce a change in one of the body's controlled condition. Some event outside the system must shut it off. If the action is not stopped, it can "run away" and be lethal.
\section{Planes :-}
\begin{figure}[h]
\centering
\includegraphics[scale=0.25]{Planes.jpg}
\end{figure}
\section{Body Cavities :-}
Body Cavities are spaces that enclose internal organs.
\linebreak


\raggedright
\raggedright
\begin{tabular}{|c|c|}
\hline
Cavity & Description \\
\hline
\textbf{Cranial Cavity} & Formed by cranial bones and contains brain. \\
\hline
\textbf{Vertebral Canal} & Formed by vertebral column and contains spinal cord\\
& and the beginnings of spinal nerves\\
\hline
\textbf{Thoracic Cavity} & Chest cavity; contains pleural and pericardial\\
& cavities and the mediastinum\\
\hline
\textit{Pleural Cavity} & A potential space between the layers of the\\
& pleura that surrounds a lung\\
\hline
\textit{Pericardial Cavity }& A potential space between the layers of the\\
& pericardium that surrounds the heart\\
\hline
\textit{Mediastinum} & Central portion of thoracic cavity between the\\
& lungs; extends from sternum to vertebral column\\ & and from first rib to diaphragm; contains heart,\\
& thymus, esophagus, trachea, and several large\\
& blood vessels.\\
\hline
\textbf{Abdominopelvic Cavity} & Subdivided into abdominal and pelvic cavities\\
\hline
\textit{Abdominal Cavity} & Contains stomach, spleen, liver, gallbladder,\\
& small intestine and most of large intestine; the\\
& serous membrane of the abdominal cavity is\\
& the peritoneum\\
\hline
\textit{Pelvic cavity} & Contains urinary bladder; portions of large\\
& intestine, and internal organs of reproduction\\
\hline
\raggedright
\end{tabular}

\begin{figure}[h]
\centering
\includegraphics[scale=0.45]{Thoracic Cavity.jpg}
\end{figure}

\section{Abdominopelvic Regions and Quadrants :- }
\begin{figure}[h]
\centering
\includegraphics[scale=0.40]{Abdominopelvic Cavity.jpg}
\end{figure}

\section{Medical Imaging :-}
Refers to  techniques and procedures used to create images of the human body.(better visualisation for diagnosing anatomical and physiological abnormalities and deviations). Medical imaging includes :-
\textbf{Radiography, Magnetic Resonance Imaging (MRI), Computed Tomography (CT, Ultrasound Scanning, Coronary (Cardiac) Computed Tomography Angiography (CCTA) Scan, Positron Emission Tomography (PET), Endoscopy and Radionuclide Scanning.}







\end{small}

\end{document}

